%!TEX root = ../thesis.tex
\section{部品の製作}
本研究で製作したロボットアームには,3Dプリント部品と板金部品を使用した.3Dプリント部品は,FDM方式の3Dプリンタを用いて製作した(図\ref{fig:3Dprint}参照).一方,板金部品については,Meviy\footnote{Meviyは,ミスミ株式会社が提供する板金・機械加工のオンライン外注サービスである.}を利用して外注製作を行った(図\ref{fig:meviy}参照).このサービスを利用することで,短納期かつ精度の高い部品を調達することができた.

その他の使用部品に関しては,既製品を採用した.特に,QDDモータ以外の部品はすべてMISUMIカタログで発注可能な汎用部品を用いた.部品選定基準は,価格,納期,および性能である.図\ref{fig:parts}に部品リストを示す.
\begin{figure}
  \centering
  \includegraphics[width=10cm]{images/product/3Dprint.jpg}
  \caption{3D printed parts}
  \label{fig:3Dprint}
\end{figure}
\begin{figure}
  \centering
  \includegraphics[width=10cm]{images/product/meviy.jpg}
  \caption{Meviy parts}
  \label{fig:meviy}
\end{figure}
\begin{figure}
  \includegraphics[width=14cm]{images/product/rist.pdf}
  \caption{Parts list}
  \label{fig:parts}
\end{figure}
\clearpage
\newpage