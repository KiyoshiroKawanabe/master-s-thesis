%!TEX root = ../thesis.tex
\section{まとめ}
本研究では,オープンプラットフォームオフィスロボット開発の第一段階として,QDDモータを用いたロボットアームの設計と製作を行った.従来のオフィスロボットが対象としている作業を調査し,「机の片づけ作業」を対象作業とした,作業を行うロボットアームに求められる要求仕様を策定し,それに基づきロボットアームの設計を行った.設計したロボットアームは6自由度を有し,すべての関節にQDDモータを採用している.設計にはAutodesk社のInventor2023を使用し,部品の構造解析を行った.また,部品外注サービスであるMeviyを活用し,短期間での部品調達と製作を可能にした.設計データや部品リストなどはすべてGitHub上で公開している.

\section{今後の展望}
今後の課題として,動作実験を通じた性能評価の実施,他のタスクにも対応可能な仕様拡張,さらには設計データのさらなる汎用性向上を検討する必要がある.また,オープンプラットフォームとしての設計データの公開を通じて,他の研究者や開発者による再利用や改良を促進し,技術の発展に寄与することを目指す.
\newpage