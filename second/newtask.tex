%!TEX root = ../thesis.tex
\section{作業調査}
本研究では,従来のオフィスロボットが対象としている作業を調査し,ロボットアームの検証に適した作業対象を検討する.本研究で開発するロボットアームは,オフィスロボットとして最低限の機能を満たすことを目的としている.そのため,アームの性能を効果的に検証するためには,従来のオフィスロボットが頻繁に対象としており,オフィス環境における基本的な作業として位置づけられるタスクを選定することが重要である.

従来のオフィスロボットが実施している作業事例を動画および文献から収集し,合計77例を抽出した.その結果,約61\%の作業が台車移動とピック\&プレースを組み合わせた作業であることが確認された.代表的な事例として,部屋の片付けや荷物の運搬が挙げられる.その他の作業事例として,ドアの開閉,ボタンの押下,フロアの巡回などが観察された.

これらの結果から,現行のオフィスロボットが主に対象としている作業は,物体を把持し移動させる作業であると推察される.また,ロボットアームの検証作業を選定するにあたっては,基本的な作業であることに加え,多様な物体操作および他の作業への応用性を考慮することが重要である.以上を踏まえ,ロボットアームの検証対象として「机の片付け作業」を選定することが適切であると判断した.

机の片付け作業は,多様な物体の把持能力が求められるだけでなく,研究者が自由に操作条件を設定しやすい特性を有している.さらに,本作業は,散乱した床の片付け作業や棚の物体操作など,他のタスクへの応用が可能であるため,ロボットアームの性能評価において有用性が高いと考える.
\newpage