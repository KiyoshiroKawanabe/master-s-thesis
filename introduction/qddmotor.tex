%!TEX root = ../thesis.tex

\section{QDDモータの可能性}
既存のオフィスロボットのロボットアームでは,高減速比のサーボモータが一般的に使用されているが,これにはバックドライバビリティが低いという欠点がある.バックドライバビリティとは,「アクチュエータや動力伝達機構において,出力節に適当な力を加えたときに,その節が可動し,かつそれが入力節側に伝わる性質」である\cite{backdrive:online}.そのため,衝突時の柔軟な応答が困難となり,人や物体へ危害を加えるリスクがある.
一方で,QDD(Quasi-Direct Dribe)モータは,高いバックドライバビリティと高トルクを両立できる特性を持ち,衝突時の安全性を向上させる可能性がある.特にオフィス環境では,人だけでなく,PCやディスプレイなど衝撃に弱い機器が多いため,QDDモータの採用は非常に有用である.しかし,QDDモータをロボットアームに活用した事例は少なく,その技術的なノウハウや応用可能性についてはまだ十分に探求されていない.
