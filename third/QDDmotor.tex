%!TEX root = ../thesis.tex
\section{QDDモータの選定}
飯塚ら\cite{飯塚浩太2021}の研究では,減速比10:1のQDDモータが使用されている.また,Zhaoら\cite{10106520}の研究においては,減速比9:1のQDDモータが採用されていた.これらの先行研究は,ロボットアームと人の衝突時の衝撃力軽減を実現するためには,この程度の減速比を持つモータを採用することが有効であることを示唆している.

さらに,本研究ではオープンプラットフォームの観点を重視し,他のロボットアームに使用されているモータと同程度,またはそれ以下の価格であることを選定基準とした.調査対象としたオフィスロボットのうち,ロボットアームに使用しているアクチュエータの価格は,約3万円から約8万円であったため,価格上限を8万円に設定した.

以上を踏まえ,本研究では,持つSteadywinのGIM8108-8とGIM3505-8を採用した.減速比は8:1であり,先行研究で使用されていたモータと同程度の減速比を持つ.また,価格は約19,000円と約18,000円であり,価格上限である8万円を下回っている.
\subsection{選定したモータの性能}
表\ref{tab:QDDComparison}に,選定したQDDモータとReachy\cite{Reachy:online}のアームに使用されているモータ(Dynamixel MX-106T)の比較を示す.選定したSteadywinのGIM8108-8とGIM3505-8は,減速比が8:1で,先行研究で使用していた減速比と近しい物になっている.適切に制御することで衝突時の衝撃を軽減することが可能である.また,許容ラジアル荷重とアキシアル荷重も高いため,追加の補強部品を必要とせずにロボットアームの構造を簡略化することが可能である.
\begin{table}[]
  \centering
  \caption{QDD motor performance comparison}
  \label{tab:QDDComparison}
  \begin{tabular}{lccc}
  \hline
               & \begin{tabular}[c]{@{}c@{}}SteadyWin\\ GIM8108-8\end{tabular} & \begin{tabular}[c]{@{}c@{}}SteadyWin\\ GIM3505-8\end{tabular} & \begin{tabular}[c]{@{}c@{}}Dynamixel\\ MX-106T\end{tabular} \\ \hline
  減速比          & 8 : 1                                                         & 8 : 1                                                         & 225 : 1                                                     \\
  定格トルク(Nm)    & 7.5                                                           & 0.65                                                          & -                                                           \\
  最大トルク(Nm)    & 22                                                            & 1.27                                                          & 10                                                          \\
  重量(g)        & 396                                                           & 97                                                            & 153                                                         \\
  無負荷回転数(rpm)  & 320                                                           & 384                                                           & 55                                                          \\
  許容ラジアル荷重(N)  & 900                                                           & 300                                                           & 40                                                          \\
  許容アキシアル荷重(N) & 225                                                           & 75                                                            & 20                                                          \\
  価格(円)        & 約19,000                                                       & 約18,000                                                       & 約80,000                                                     \\ \hline
  \end{tabular}
\end{table}

\subsection{QDDモータの欠点}
QDDモータは高トルク・高スピードでの動作が可能であるという利点を持つ一方で、この特性は制御が不十分な場合にリスクとなる.特に、通電時に動作が暴走した場合、一般的なモータと比較して周囲の人や機器への被害が大きい.そのため、QDDモータを安全に運用するためには、適切な制御と安全対策が必要である.
\newpage