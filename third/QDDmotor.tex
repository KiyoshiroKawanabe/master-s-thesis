%!TEX root = ../thesis.tex
\section{QDDモータの選定と特徴}
採用したQDDモータは2種類あり,肩関節と肘関節にトルクの大きいQDDモータ(Steadywin GIM8108-8)を使用し,手首およびエンドエフェクタにはトルクの小さいモータ(Steadywin GIM3505-8)を使用する.QDDモータを採用することで,柔軟な関節動作を実現し,衝突時の安全性を向上させることが可能となる.以下では,肩関節と肘関節に使用するトルクの大きいモータを例に,一般的なロボットアーム用モータと比較してその特性を論じる.
\subsection{QDDモータと一般的なモータの比較表}
表\ref{tab:QDDComparison}に,本研究で使用するQDDモータと一般的なロボットアーム用モータ(Dynamixel MX-106T)の比較を示す.
\begin{table}[h]
  \centering
  \begin{tabular}{lcc}
    \hline
    \multicolumn{1}{c}{\textbf{}} & \begin{tabular}[c]{@{}c@{}}SteadyWin\\ GIM8108-8\end{tabular} & \begin{tabular}[c]{@{}c@{}}Dynamixel\\ MX-106T\end{tabular} \\ \hline
    減速比                           & 8:1                                                           & 225:1                                                      \\ 
    最大トルク (Nm)                     & 22                                                            & 10                                                         \\ 
    モータ重量 (g)                      & 396                                                           & 153                                                        \\ 
    無負荷回転数 (rpm)                   & 320                                                           & 55                                                         \\ 
    許容ラジアル荷重 (N)                   & 900                                                           & 40                                                         \\ 
    許容アキシアル荷重 (N)                  & 225                                                           & 20                                                         \\ \hline
  \end{tabular}
  \caption{QDDモータと一般的なロボットアーム用モータの比較}
  \label{tab:QDDComparison}
\end{table}

\subsection{減速比}
採用した QDD モータの減速比は約 8:1 であり,これは一般的なロボットアームに使用されるモータの減速比(225:1)に比べて著しく低い.低減速比により,バックドライバビリティが高まり,柔軟な関節動作を実現することが可能となる.これにより,外力に対して柔軟に対応できるため,衝突時の衝撃を最小限に抑えられる.また,柔軟性の向上により,安全性が求められるオフィス環境での運用に適している.
\subsection{高トルクと重量}
採用した QDD モータは,最大トルクが 22Nm であり,一般的なロボットアーム用モータの約 2.2 倍のトルクを提供できる.これにより,肩関節や肘関節において高い負荷がかかる作業を可能にする.しかし,QDD モータは重量が 396g と,一般的なモータ(153g)の約 2.6 倍であり,設計では軽量化が求められる.
\subsection{無負荷回転数}
QDD モータの無負荷回転数は 320rpm であり,これは一般的なロボットアーム用モータの 55rpm に比べて約 6 倍高速である.高速な回転数は,動作の迅速化や応答性の向上に寄与する.特に,短い時間での作業遂行が求められる環境での利便性を高める.
\subsection{許容荷重}
QDD モータは,ラジアル荷重とアキシアル荷重の許容値が非常に高い.ラジアル荷重の許容値は 900N,アキシアル荷重は 225N であり,これは一般的なモータ(ラジアル荷重 40N,アキシアル荷重 20N)に比べて大幅に優れている.これにより,モータの出力軸に直接部品を接続する片持ち梁設計が可能となる.片持ち梁設計は,追加の支持構造が不要となるため,設計の簡素化や部品点数の削減が期待できる.
\subsection{QDDモータの欠点}
QDDモータは高トルク・高スピードでの動作が可能であるという利点を持つ一方で、この特性は制御が不十分な場合にリスクとなる.特に、通電時に動作が暴走した場合、一般的なモータと比較して周囲の人や機器への影響が大きい.そのため、QDDモータを安全に運用するためには、動作制御における適切な安全対策が不可欠である.また.QDDモータは重量が大きいため,設計時には軽量化を考慮する必要がある.軽量化を図ることで,暴走時のリスク軽減にも繋がる.
\newpage