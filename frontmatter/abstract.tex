%!TEX root = ../thesis.tex
\chapter*{概要}
\thispagestyle{empty}
%
\begin{center}
  \scalebox{1.5}{オープンプラットフォームオフィスロボットの開発}\\
  \scalebox{1.2}{-QDDモータを使用したロボットアームの設計と製作-}\\
\end{center}
\vspace{1.0zh}
%

本論文では,QDDモータを用いたオフィスロボットのロボットアームのメカニズム設計および製作について述べる.近年,人手不足を背景に,オフィス業務の効率化を目的としたロボット技術の導入が注目されている.多くの企業や研究室においてオフィスロボットの開発が進められているが,既存のロボットは設計データが公開されていないものが多く,標準的なプラットフォームが不足している.こうした状況に対し,本研究では,オープンプラットフォーム開発の,第一段階としてQDDモータを用いたロボットアームのメカニズムを設計し,その設計データを公開することを目的とする.

オフィス環境において,ロボットは人や物などに対する安全性が求められる.QDDモータは低減速比による高応答性を持ち,適切な制御を施すことで外力に対して柔軟に動作可能な関節を実現できる.本研究では,オフィスロボットの代表的なタスクとして「机の片づけ作業」を対象とし,ロボットアームの要求仕様を設定した.
設計したロボットアームは6自由度を有し,すべての関節にQDDモータを採用している.また,部品は外注製作に対応しており,加工を必要とせず,組み立てのみで再現可能である.

今後の課題として,動作実験を通じた性能評価の実施,他のタスクにも対応可能な仕様拡張,さらには設計データのさらなる汎用性向上を検討する必要がある.


キーワード: オフィスロボット,ロボットアーム,QDDモータ,オープンプラットフォーム
%
\newpage
%%
\chapter*{abstract}
\thispagestyle{empty}
%
\begin{center}
  \scalebox{1.3}{title}
\end{center}
\vspace{1.0zh}
%


keywords: 
