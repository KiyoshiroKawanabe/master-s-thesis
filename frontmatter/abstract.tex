%!TEX root = ../thesis.tex
\chapter*{概要}
\thispagestyle{empty}
%
\begin{center}
  \scalebox{1.5}{オープンプラットフォームオフィスロボットの開発}\\
  \scalebox{1.2}{-QDDモータを使用したロボットアームの設計と製作-}\\
\end{center}
\vspace{1.0zh}
%
 本論文では,オープンプラットフォームなオフィスロボット開発の第一段階として,ロボットアー
ムの設計・製作について述べる.オフィスロボットの明確な定義は現状存在しないため,本研究では
「オフィス環境で活動するモバイルマニピュレータロボット」と定義する.近年,PALロボティク
ス社のTIAGo,トヨタのHSR,Pollenロボティクス社のReachyなど,様々なオフィスロボットが
開発されているものの,ハードウェアのオープンプラットフォーム化が進んでいない点が課題とし
て挙げられる.ハードウェアのオープンプラットフォームが普及することで,分野の活性化に寄与
すると感がられる.そこで本研究では,オープンプラットフォームなオフィスロボットの開発を
目指し,その第一段階として,ロボットアームの開発に取り組んだ.\\
 オフィスロボット用ロボットアームの開発にあたっては,まずオフィス環境で求められる作業
分析し,具体的な作業内容を明確化する必要があった.既存のオフィスロボット(前出のロボット
を含む20台)を対象に動画及び文献調査を行い,80種類の作業を抽出し分析した.その結果,片
付け作業が多く行われていることが判明し,本研究では「デスクの片付け」を作業タスクとして
設定した.この作業は,多様な形状・大きさの物体を扱う必要があるため,ロボットアームには
高い汎用性が求められる.また,人とロボットが協働するオフィス環境においては,安全性も重
要な要素となる.これらの点を踏まえ,ロボットアームの要求仕様を策定し設計を行った.\\
 開発したロボットアームは,全ての関節にQDDモータを採用している点が最大の特徴である.
QDDモータの高いバックドライバビリティにより,外力が加わった際に関節が柔軟に動作し,衝撃
を吸収することが可能となる.これにより,従来の減速機付きモータを採用したアームと比較し
て,人や物体との衝突時の安全性が向上する.さらに,QDDモータの堅牢性を活かした設計によ
り,アームはシンプルな構造で実現されている.具体的には,部品点数を少なく抑え,組み立て
を容易にすることで,オープンプラットフォームとして重要な再現性と拡張性を高めている.
これにより,設計データを公開することで,他の開発者による容易な複製と,必要に応じたカス
タマイズが可能となる.本研究の成果は,将来的なオープンプラットフォームオフィスロボット
の開発基盤となることが期待される.

キーワード: オフィスロボット ロボットアーム QDDモータ オープンプラットフォーム
%
\newpage
%%
\chapter*{abstract}
\thispagestyle{empty}
%
\begin{center}
  \scalebox{1.3}{title}
\end{center}
\vspace{1.0zh}
%


keywords: 
