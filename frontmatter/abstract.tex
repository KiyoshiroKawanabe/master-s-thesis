%!TEX root = ../thesis.tex
\chapter*{概要}
\thispagestyle{empty}
%
\begin{center}
  \scalebox{1.5}{オープンプラットフォームオフィスロボットの開発}\\
  \scalebox{1.2}{-QDDモータを使用したロボットアームの設計と製作-}\\
\end{center}
\vspace{1.0zh}
%

本研究では,オープンプラットフォームオフィスロボット開発の第一段階として,QDDモータを使用したロボットアームの設計と製作を行った.近年,人手不足を背景に,オフィスロボットの実用化が期待されている.様々な企業や研究室で開発が行われているが,それらのロボットは設計データを公開しておらず,オープンプラットフォームが不足している.本研究では,この課題を解決する第一歩として,QDDモータを使用したオフィスロボットのロボットアームのメカニズムを設計・製作し,その設計データを公開することを目的とする.QDDモータは,低減速比による高い応答性を備えており,適切に制御することで衝突時の衝撃を軽減できることが先行研究により確認されている.本研究では,人に被害を加えないことが極めて重要なオフィス環境において,この特性を活用する.

従来のオフィスロボットの作業内容を調査し,代表的なタスクである「机の片づけ作業」を対象に,ロボットアームの要求仕様を策定した.設計にはAutodesk社のInventor2023を使用し,部品の構造解析を行いながら6自由度のロボットアームを開発した.また,部品外注サービスを活用し,短期間での部品調達と製作を可能にした.ロボットアームの設計データおよび部品リストはGitHub上で公開され,他の開発者による利用および改良が可能である.
これにより,オフィスロボット開発のコスト削減やスピード向上を促進し,また,QDDモータの新たな応用可能性を切り拓くプラットフォームの構築に貢献する


キーワード: オフィスロボット,ロボットアーム,QDDモータ,オープンプラットフォーム
%
\newpage
%%
\chapter*{abstract}
\thispagestyle{empty}
%
\begin{center}
  \scalebox{1.3}{title}
\end{center}
\vspace{1.0zh}
%


keywords: 
