%!TEX root = ../thesis.tex
\chapter*{概要}
\thispagestyle{empty}
%
\begin{center}
  \scalebox{1.5}{オープンプラットフォームオフィスロボットの開発}\\
  \scalebox{1.2}{-QDDモータを使用したロボットアームの設計と製作-}\\
\end{center}
\vspace{1.0zh}
%
本論文は,オープンプラットフォームオフィスロボットの開発を目的として,ロボットアームの設計と製作について述べる.オフィスロボットは,PAL Robotics社のTIAGoやトヨタのHSRのように多くの事例が存在するが,ハードウェアのオープンプラットフォーム化が進んでいないことが課題である.オープンプラットフォーム化により,設計データの公開を通じて研究開発の活性化が期待される.本研究では,オフィス環境におけるロボットアームの作業として「机の片付け」を設定し,その要求仕様を策定した.分析の結果,ロボットアームには高い汎用性と安全性が求められることが明らかとなった.これに基づき,すべての関節にQDDモータを採用したロボットアームを設計した.QDDモータは高いバックドライバビリティを有し,衝突時の柔軟な応答を可能にすることで安全性を向上させる特性を持つ.さらに,部品点数を減らし,組み立てを容易にする設計を行ったことで,オープンプラットフォームとしての再現性と拡張性を高めた.製作したロボットアームは,要求仕様を満たすとともに,設計データを公開することで他の研究者や開発者による再利用や改良を可能にする.本研究は,オープンプラットフォームオフィスロボットの基盤となることを目指しており,QDDモータの新たな応用例としても意義がある.

キーワード: オフィスロボット,ロボットアーム,QDDモータ,オープンプラットフォーム

%
\newpage
%%
\chapter*{abstract}
\thispagestyle{empty}
%
\begin{center}
  \scalebox{1.3}{title}
\end{center}
\vspace{1.0zh}
%


keywords: 
